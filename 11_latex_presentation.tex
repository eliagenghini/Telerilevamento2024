% LA MIA PRIMA PRESENTAZIONE IN LATEX

\documentclass{beamer} % Tipo di documento. Beamer è un pacchetto Latex x creare presentazioni
\usepackage{graphicx} % Required for inserting images

% TEMA E VARIAZIONE COLORI: 
% https://mpetroff.net/files/beamer-theme-matrix/  per vedere temi e colori disponibili
\usetheme{Frankfurt} % Altri temi carini: Bergen, Montpellier
\usecolortheme{spruce} % Altri colori carini utilizzabili: crane, seahorse, whale

\title{Presentazione in LaTeX} % Per mettere il titolo della presentazione
\author{Irene Altea De Vincenzo} % Per mettere l'autore
\date{May 2024} % Per mettere la data

\begin{document} % Per iniziare il documento

\maketitle % Crea slide iniziale con le informazioni fornite

% --- Creare un comando che viene esequito all'inizio di ogni sezione --- 

% \AtBeginSection[]: Questo comando viene eseguito all’inizio di ogni sezione --> quando inizia una nuova sezione, il codice all’interno delle parentesi graffe { ... } verrà eseguito.

% \begin{frame}{Outline}: Questo comando crea una nuova slide con il titolo “INDICE”. Il testo tra parentesi graffe verrà visualizzato come titolo della slide.

% \tableofcontents[currentsection]: Questo comando genera un elenco delle sezioni della presentazione. L’opzione currentsection indica che verranno elencate solo le sottosezioni della sezione corrente.

\AtBeginSection[] 
{
\begin{frame}{INDICE}
\tableofcontents[currentsection]
\end{frame}
}

\section{Introduzione} % Definisco nuova sezione

\begin{frame}{My first slide} % Inserisco nuova slide e metto il titolo slide
    Qui puoi scrivere il testo che vuoi mettere nella slide
\end{frame} % Chiudo slide

\begin{frame}{ Facciamo un elenco puntato}
In questa slide proverò a fare il mio primo elenco puntato
    \begin{itemize} % Inizio elenco 
        \item Punto 1 % Aggiungo un primo punto
        \item \pause Punto 2 % Aggiungo un secondo punto etc..
        \item \pause Punto 3
    \end{itemize} % Chiudo elenco puntato
\end{frame} % Chiudo slide

\begin{frame}{Cambiamo la dimensione del testo}
    \scriptsize{Scriviamo un testo in piccolo} % \scriptsize{} imposta una dimnesione del testo ridotta
    
    \huge{Scriviamo il testo in grande} % \huge serve per scrivere il testo grande
\end{frame}

\begin{frame}{Cambiamo la tipologia di testo}
    Cambiamo il tipo di testo 
    \bigskip % Serve a lasciare un grande spazio tra una riga e l'altra
    
    Cambiamo il \textbf{tipo} di testo % \textbf{} --> grassetto
    \bigskip
    
    Cambiamo il \textit{tipo} di testo % \textit{} --> corsivo 
\end{frame}

\section{FORMULE} % Apriamo nuova sezione

\begin{frame}{Formule }
Formula per calcolare la deviazione standard
\bigskip
    \begin{equation}
        \delta = \sqrt{\frac{\displaystyle\sum_{i=1}^{N}{(x - \mu)^2}}{N}}
    \end{equation}
\end{frame}

\section{Risultati}

\begin{frame}{Risultati raggiunti}
   \begin{figure}
       \centering
       \includegraphics[width=0.5\linewidth]{patata1.jpg}
       \caption{Sono un patata a programmare}
       \label{fig:patata1}
   \end{figure}
\end{frame}

\begin{frame}{Confronto}
    \begin{figure}
        \centering
        \includegraphics[width=.4\linewidth]{patata2.jpg}
        \includegraphics[width=.4\linewidth]{patata1.jpg}
        \caption{Irene prima vs dopo}
        \label{confronto}
    \end{figure}
\end{frame}

% FARE COLONNE
\begin{frame}{Fare colonne}
\begin{column} {0.5\textwidth}
    Posso scrivere una piccola parte di testo in colonna
\end{column}

\bigskip

        \begin{column}{0.5\textwidth}
        Guardate che patata felice. 
        \includegraphics[width=0.4\linewidth]{patata2.jpg} % Per allineare un'immagine al fianco della colonna
        \end{column}

\end{frame}

\end{document}
