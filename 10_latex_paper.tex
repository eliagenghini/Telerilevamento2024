% COME SCRIVERE UN PAPER CON LATEX

% Per i commenti
% In LaTeX le funzioni iniziano con \ e sono contenute all'interno di parentesi graffe {}

\documentclass[a4paper, 12pt]{article} % \documentclass{nometemplate} --> Indica il tipo di documento che andremo a creare
\usepackage{graphicx} % Pacchetto necessario per inserire immagini
\usepackage{lineno} % Per aggiungere numeri di linea a un documento latex
\usepackage{hyperref} % Per creare collegamenti ipertestuali
\usepackage{natbib} % Espansione pacchetto riferimenti
\usepackage{color} % Per usare colori nel tuo testo \textcolor{red}{This text is red.}
\linenumbers % Numera ogni riga
\linespread{1.5} % Cambia l'interlinea del documento

% Ora inseriamo i primi dati:

\title{Il mio primo documento in LaTeX} % Mi fa aggiungere un titolo
\author{Irene Altea De Vincenzo} % Imposto l'autore
\date{07 Maggio 2024} % Imposto la data. Se non scrivo nulla Overleaf prenderà la data nel computer

\begin{document} % Inizio documento

\maketitle % Copia tutta la parte iniziale del documento, cioè l'Header, impostando la pagima iniziale
\nolinenumbers % Rimuove la numerazione fastidiosa per ogni riga del documento

$^1$ UNIBO - via Vattela a Pesca 3. Dipartimento del " Ma io volevo solo leccar sassi"

\tableofcontents % Crea l'indice

\begin{abstract} % Mi crea un abstract per il documento
La storia è strana ed improbabile e trova sempre il modo per sorprenderti.  
\end{abstract} % Chiude l'abstract

\section{Introduction} % Apre un paragrafo e numera il titolo. \section*{Introduction} --> L'asterisco NON fa mettere il numero
\tb La vita è davvero piena di sorprese e situazioni imprevedibili, vero?A volte sembra che la realtà stessa stia scrivendo una sceneggiatura strana e avventurosa.

%------------------ COME INIZIARE UNA SEZIONE---------------------
\section{ASSURDE COINCIDENZE STORICHE}
\subsection{CESARE E I PIRATI} % Apre un sotto-paragrafo numerato
\textbf{75ac}. Dei pirati abbordano una galera romana e rapiscono un giovane (Figure \ref{fig:enter-label}, chiedendo un riscatto di 20 talenti.
Il giovanotto, invece di spaventarsi, inizia a ridere e suggerisce ai pirati di chiedere almeno 50 talenti, poichè \textit{“Lui non vale mica così poco!"} 
I pirati lo prendono per scemo e lo deridono.
Giunti a terra per ricevere il riscatto, il giovane continua a scherzare, minacciando di farli crocifiggere tutti una volta libero. I pirati ridono ancora di più, prendono i soldi e se ne vanno.

\noindent Peccato che... era \textcolor{red}{Giulio Cesare}: il giorno dopo si presenta con una flotta al seguito nello stesso posto in cui era stato rapito, li cattura tutti e li fa crocifiggere.

\noindent Sad story but good. 

%\textbf{} --> Mette le parole in grassetto
%\textit {} --> Il testo dentro le parentesi diventa in corsivo
% \noindent rimuove l'indentazione della prima riga del paragrafo corrente
% \url{} --> Inserisce il link indicato tra parentesi nel testo
% \newpage --> crea una nuova pagina

% --------------------- COME AGGIUNGERE ELENCO PUNTATO ---------------------
\noindent Questa storia ci insegna che:
\begin{itemize} % Per iniziare elenchi puntati
    \item Se vuoi rapire qualcuno è sempre meglio informarsi su chi sia; % \item per aggiungere un punto
    \item Mai rompere il cazzo a Giulio Cesare;
    \item Se rompi il cazzo a Giulio Cesare, almeno sii abbastamza furbo da scappare lontano.
\end{itemize} % Chiude sezione puntata

% --------------------- COME AGGIUNGERE ELENCO NUMERATO ---------------------
\noindent Ulteriori insegnamenti:
\begin{enumerate} % Per iniziare elenchi numerati
    \item Giovane non vuol dire stupido; % \item per aggiungere una nuova voce all'elenco
    \item Forse era meglio scegliere un lavoro onesto;
\end{enumerate} % Chiude sezione numerata

\subsection{QUANDO LA SFIGA TI PERSEGUITA}
\textbf{1°Guerra mondiale}. I tedeschi amavano la tattica del "camuffaggio": ridipingevano le navi per farle assomigliare a quelle nemiche, così da potersi avvicinare abbastanza da poterle affodnare. 
\noindent La "Trafalgar" tedesca fu quindi traversita da "Carmanian" (inglese). La finta nave inglese, appena ridipinta, prende il mare, alla ricerca di nemici da affondare. 

\noindent Ma indovinate un po' qual è la prima nave che incontrò? LA VERA CARMANIAN!!!... che ovviamente la affondò immediatamente.

\noindent Il tutto farebbe già ridere così...se non fosse che la Carmanian (quella vera), era a sua volta, travestita da Trafalgar!

Insomma: tedeschi travestiti da inglesi, inglesi travestiti da tedeschi...chissà che cavolo gli è passato per la testa mentre si affondavano a vicenda. 

\subsection{COME SCRIVERE LE EQUAZIONI}
\subsubsection{Torniamo alle cose serie, che qui bisogna imparare a scrivere le equazioni}
Guarda quanto è scritta carina questa equazione \ref{eq:newton} %\ref serve a fare rifetimenti all'interno del testo, una volta che li hai etichettati con \label

% --------------------- COME AGGIUNGERE EQUAZIONI ---------------------
% \begin{equation} e \end{equation} --> Permette di creare un'equazione
% _1--> Il numero dopo l'underscore lo rende un pedice 
% ^2 --> Il cappellino trasforma il numero in un apice
% \times --> Aggiunge il simbolo della moltiplicazione
% \sqrt[numero] --> Mette tutto sotto una radice quadrata.
% \sqrt[3]{} --> Se voglio fare una radice cubica devo scrivere il 3 all'interno delle parentesi quadre
% \sum P1 --> Simbolo della sommatoria per la variabile P1

\begin{equation}
    F = \sqrt[3]{G \times \frac{m_1 \times m_2}{r^2}}
    \label{eq:newton}
\end{equation}

\section{Risultati}
Okay forse, nonostante tutto, sto Latex è fattibile.

\section{Discussion}
Non importa quanto sei bravo e ti impegni nella vita, se Dio ce l'ha con te, stai certo che ti farà il culo.
\citep{Dioburlone}.

% --------------------- COME INSERIRE LA BIBLIOGRAFIA --------------------- 
%\bibitem[]{}  --> Aggiunge una citazione bibliografica nel database. {} Quello che c'è nelle parentesi graffe funge da Etichetta. [] Quello che c'è tra parentesi quadre è quello che apparirà nel testo (quando andiamo a citare la bibliografia)
% \cite{Dio burlone} --> Aggiunge una citazione bibliografica al testo. Tra le parentesi graffe inseriamo l'Etichetta e come risultato otteniamo quello che abbiamo messo nelle parentesi quadre della funzione \bibitem
\begin{thebibliography}{999}
    \bibitem[Dio è un burlone,2024]{Dioburlone}
    Cristiani et alii, (2024). ‘Quell'infame mi ha fregato'. Paradise's journal, (prayer ediction).
\end{thebibliography}

%--------------------- COME INSERIRE IMMAGINE ---------------------
\begin{figure} % Inserisce immagine
    \centering % Centra immagine
    \includegraphics[width=0.5\linewidth]{cesare01.jpg} % Grandezza immagine
    \caption{Immagine Giulio Cesare} % Aggiungi testo sotto foto 
    \label{fig:enter-label} % Etichetta per richiamare l'immagine
\end{figure}
\end{document}
